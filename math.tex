\chapter{Maths}
\section{Equation formats}
\section{align}
Align to the $=$
\begin{align*}
	a&=b\\
	\leadsto caoeuuu&=d
\end{align*}
Something like this looks confusing
\begin{align*}
	&a=b\\
	&\leadsto aoeuou=d
\end{align*}
\section{Integrals}
Needs fix. Need to put the right negative space to make it look right. Put a grid in background in some way.
\subsection{Simple integral}
\begin{dgroup}[]
	\begin{dmath}[]
		\int_{}^{}\td x \, x=x^{2}
	\end{dmath}
	\begin{dmath}[]
		\int_{0}^{1}\! \td x\, x=1
	\end{dmath}
	\begin{dmath}[]
		\int_{}^{}\td x \, x=x^{2}
	\end{dmath}
	\begin{dmath}[]
		\int_{0}^{10}\! \td x\, x=100
	\end{dmath}
	\begin{dmath}[]
		\int_{}^{}\td x \, x=x^{2}
	\end{dmath}
	\begin{dmath}[]
		\int_{0}^{100}\! \td x\, x=100
	\end{dmath}
\end{dgroup}
\newpage
\subsection{With fractions}
Use thick space between two roman variables or between a roman variable and a fraction.
\begin{dgroup*}[]
	\begin{dmath}[]
		\int_{}^{}\td^{3}\mathbf{r}\;\mathbf{j}\left( \mathbf{r},t \right)=\int_{}^{}\td^{3}\mathbf{r}\; \frac{1}{2m}\left[ \psi^{*}\left( -i\hbar\nabla \right)\psi+\psi\left( i\hbar\nabla \right)\psi^{*} \right]
	\end{dmath}
	%\begin{dmath}[]
	%	\delimitershortfall=-1pt
	%	\int_{}^{}\td^{3}\mathbf{r}\;\mathbf{j}\left( \mathbf{r},t \right)=\int_{}^{}\td^{3}\mathbf{r}\; \frac{1}{2m}\left[ \psi^{*}\left( -i\hbar\nabla \right)\psi+\psi\left( i\hbar\nabla \right)\psi^{*} \right]
	%\end{dmath}
	\begin{dmath}[]
			\int_{}^{}\td^{3}\mathbf{r}\;\mathbf{j}\left( \mathbf{r},t \right)=\int_{}^{}\td^{3}\mathbf{r}\; \frac{1}{2m}\left[ \psi^{*}\left( -i\hbar\nabla \right)\psi+\psi\left( i\hbar\nabla \right)\psi^{*} \right]
	\end{dmath}
\end{dgroup*}

\subsection{Multiple integrals}
\begin{dmath}[]
	\iiint\limits_{V}\td V\,\boldsymbol{\mathrm\nabla}\cdot F = \iint\limits_{S}\td S\, F
\end{dmath}
\section{Derivatives}
\begin{dgroup}[]
	\begin{dmath}[]
		\dd{}{x}x^{2}=x
	\end{dmath}
\end{dgroup}
\section{Functions}
Examples of separations and their importance.
\begin{dgroup}[]
	\begin{dmath}[]
		\sin 2\pi\cos\theta = 0
	\end{dmath}
	\begin{dmath}[compact]
		\sin\left( \theta x \right)=\sin\theta x\neq \sin\theta \, x=\sin\left( \theta \right)x
	\end{dmath}
\end{dgroup}
\section{Multiplications}
\begin{dgroup}[]
	\begin{dmath}[]
		1\cdot 1
	\end{dmath}
	\begin{dmath}[]
		1 \! \cdot\! 1
	\end{dmath}
	\begin{dmath}[]
		1\cdot 1
	\end{dmath}
	\begin{dmath}[]
		0\cdot 0
	\end{dmath}
	\begin{dmath}[]
		x\cdot x
	\end{dmath}
	\begin{dmath}[]
		\boldsymbol{\mathrm\nabla} \cdot x
	\end{dmath}
	\begin{dmath}[]
		x \!\cdot\! x
	\end{dmath}
	\begin{dmath}[]
		\boldsymbol{\mathrm\nabla}\!\cdot\! x
	\end{dmath}
\end{dgroup}
\section{Hyphen}
\section{Braces}
I recommend using dynamic sized braces like $\left(  \right)$. However when two nested braces have the same size the readability is increased if the exterior brace has a bigger size. It is very important if the nested braces are just next to each other. One possibility is to add a vphantom and don't forget to put a $\!$ if needed.
\begin{align}
 \int \td^3 \vec r \; \vec j(\vec r , t)  &= \int \td^3\vec r \; \frac{1}{2m} \left[ \!\vphantom{A^{A}}
 \braket{\psi, t }{\vec r} \sand{\vec r}{\vec p}{ \psi, t}
 + 
 \sand{ \psi, t}{\vec p}{\vec r} \braket {\vec r}{\psi, t }
 \right] 
\end{align} 
Without the vphantom:
\begin{align}
 \int \td^3 \vec r \; \vec j(\vec r , t)  &= \int \td^3\vec r \; \frac{1}{2m} \left[ 
 \braket{\psi, t }{\vec r} \sand{\vec r}{\vec p}{ \psi, t}
 + 
 \sand{ \psi, t}{\vec p}{\vec r} \braket {\vec r}{\psi, t }
 \right] 
\end{align} 
Alternative with %\delimitershortfall=-1pt and \mathinner


\section{Prime}
$A'$ and $A^{\prime}$ look exactly the same.
\section{Square root}
\section{Matrix}
When there is an underbrace under the matrix, one needs a $\!$ on both sides
\begin{dgroup*}[]
	\begin{dmath}[]
		\underbrace{\!
			\begin{pmatrix}
				1&2\\
				3&4
			\end{pmatrix}
		\!}_{blabla}
		=
		\begin{pmatrix}
			5&6\\
			7&8
		\end{pmatrix}
	\end{dmath}
	\begin{dmath}[]
		\underbrace{
			\begin{pmatrix}
				1&2\\
				3&4
			\end{pmatrix}
		}_{blabla}
		=
		\begin{pmatrix}
			5&6\\
			7&8
		\end{pmatrix}
	\end{dmath}
\end{dgroup*}
\begin{dmath}[]
	\begin{pmatrix}
		\hdotsfor{3}\\
		\ldots&1&\ldots\\
		\hdotsfor{3}
	\end{pmatrix}
\end{dmath}
\section{Miscellaneous}
\begin{enumerate}[i)]
	\item Use $\ell$ instead of $l$. Increases readability.
	\item $\xrightarrow{T}$ instead of $\stackrel{T}{\rightarrow}$
	\item $\ddots$
\end{enumerate}

