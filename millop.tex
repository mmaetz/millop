%%%\documentclass[paper,notoc,12pt]{JHEP3}
%%%mmaetz: put a4paper
\documentclass[a5paper]{book}
%%%\documentclass[12pt,a4paper,openany]{memoir}
%%%mmaetz: reduce margins
%%%mmaetz: omitted temporarily to have the margins for the change remarks.
\usepackage[a5paper,left=2cm, right=1cm,top=1.5cm, bottom=1.5cm]{geometry}
%mmaetz: use tikz package
\usepackage{tikz,pgflibraryshapes}
%\usepackage[headings,cm]{fullpage}
\usetikzlibrary{positioning,calc,matrix,chains,scopes,fit,decorations,decorations.pathmorphing,decorations.pathreplacing,arrows,patterns}
\usepackage[utf8]{inputenc}

\usepackage{epsfig,cite,amsmath}
\usepackage{amsbsy} 
\usepackage{graphicx}
%\usepackage{axodraw}
\usepackage{pstricks}
\usepackage{color}
\usepackage{latexsym}
%%% mmaetz packages
%%% Theorems like Theorem 3.1, proof environment
\usepackage{amsthm}
%%% Using this packages numbers with units are always typed correctly.
\usepackage{siunitx}
%%% To cross out some stuff.
\usepackage{cancel}
%%% To have an enumerate environment with options to get like.
%%% i) bla
%%% ii) lala
%%% typing just \item
\usepackage{enumerate}
%%% To use \bm\mathrm instead of \vec one needs the appropriate greek letters
\usepackage{upgreek}
%%% bold math
\usepackage{bm}
%%% Nice tabular
\usepackage{booktabs}
%%% For delimgrowth, have nice nested braces. Doesn't work well with AMSmath. :/
%\usepackage{nath}
%\delimgrowth=1
\usepackage{amsbsy}
%%% To change the headers. (Coming soon)
\newcommand{\changefont}{%
	\fontsize{9}{11}\selectfont
}
\usepackage{fancyhdr}
\pagestyle{fancy}
\fancyhead[LE]{\changefont\slshape \rightmark} %section
\fancyhead[RE]{\thepage}
\fancyhead[RO]{\changefont\slshape \leftmark} % chapter
\fancyhead[LO]{\thepage}
\fancyfoot[C]{}

%%% To use mathclap.
\usepackage{mathtools}
%%% mmaetz: This is for the change notes
%\usepackage[deletedmarkup=none]{changes}
%%% Use this option instead of the above one to make the red stuff disappear and the blue stuff become black. (And also remove the footnotes done with the change package.)
%%% Below are the options used for the change package. Note that the default is to cross out the deleted stuff but this doesn't work well in a math environment so the default settings have been changede.
%\usepackage[deletedmarkup=none,final]{changes}
%\setdeletedmarkup{\textcolor{red!75!black}{#1}}
%\setauthormarkupposition{left}
%\setremarkmarkup{\footnote{#1: \textcolor{Changes@Color#1}{#2}}}
%\setremarkmarkup{\marginpar{#1:#2}}
%%% mmaetz: To suppress the change notes put the final option like that:
%\usepackage[final]{changes}
%%% mmaetz: I'm using an authors id
%\definechangesauthor[name={Marc Maetz},color=blue!50!black]{MM}
%\definechangesauthor[name={Peter Strassmann},color=red!50!black]{PS}

%%% mmaetz: Can be useful for something but forgot what and I'm not using it here.
%\usepackage{etoolbox}
%%% mmaetz: Just discovered breqn which provides automatic line breaking. Very nice, more powerful but requires a bit of getting used to it.
%%% flexisym is needed by breqn
\usepackage{flexisym}
\usepackage{breqn}
%%% To make the references clickable
%%% The color settings are in the tikzstuff.tex file.
\usepackage{hyperref}

%\hypersetup{colorlinks=false,linkbordercolor=gray}

%%% mmaetz: See this as a suggestion. This can be changed back any time.
\renewcommand{\vec}[1]{\bm{\mathrm#1}}
%\renewcommand{\vec}[1]{\mathbf#1}
%\StrLeft{#1}{8}
%\renewcommand{\vec}[1]{\IfStrEqCase{\StrLeft{#1}{8}}{{\mathcal}{\bm{#1}}}[\bm{\mathrm#1}]}
%\newcommand{\bvec}[1]{\bm{#1}}
%%%
\newcommand{\s}{\ensuremath{\;}}
\newcommand{\lag}{\ensuremath{\cal L}}
\newcommand{\dslash}{\ensuremath{\displaystyle{\not}}}
\newcommand{\Ta}{\ensuremath{\tilde T}}

%\newcommand{\ket}[1]{\ensuremath{\left| {#1} \right>}}
%\newcommand{\bra}[1]{\ensuremath{\left< {#1} \right|}}
%\newcommand{\braket}[2]{\ensuremath{\left< {#1} \, \right| \left. {#2} \right>}}
%\newcommand{\ketbra}[2]{\ensuremath{ {\ket{#1} \bra{#2}}}}
%\newcommand{\proj}[1]{\ensuremath{ {\ket{#1} \bra{#1}}}}
%\newcommand{\sumproj}[1]{\ensuremath{ {\sum_{#1}\proj{#1}}}}
%\newcommand{\sand}[3]{\ensuremath{\left< {#1} \right|{#2} \left|{#3}\right>}}
%mmaetz: breqn doesn't like \left< and \right> so I rewrite your macros:
\newcommand{\ket}[1]{\ensuremath{\left\lvert {#1} \right\rangle}}
\newcommand{\bra}[1]{\ensuremath{\left\langle {#1} \right\rvert}}
%mmaetz: Added negative thin space, added vphantom (makes both sides of the braket of the same size in all cases)
\newcommand{\braket}[2]{\ensuremath{\left\langle {#1}\vphantom{#2} \, \right\rvert \left.\! {#2}\vphantom{#1} \right\rangle}}
\newcommand{\ketbra}[2]{\ensuremath{ {\ket{#1} \bra{#2}}}}
\newcommand{\proj}[1]{\ensuremath{ {\ket{#1} \bra{#1}}}}
\newcommand{\sumproj}[1]{\ensuremath{ {\sum_{#1}\proj{#1}}}}
\newcommand{\sand}[3]{\ensuremath{\left\langle {#1}\vphantom{#3} \right\rvert{#2} \left\lvert{#3}\vphantom{#1}\right\rangle}}


\newcommand{\vevj}[2]{\left\langle {#1} \right\langle_{#2} }
\newcommand{\vev}[1]{\left\langle {#1} \right\rangle }


\newcommand{\comm}[2]{\left[ {#1}, {#2} \right] }
\newcommand{\acomm}[2]{\left\{ {#1}, {#2} \right\} }

\newcommand{\norm}[1]{\left\lvert {#1} \right\rvert }

%%% mmaetz: For the differential.
\newcommand{\td}{\mathrm d }

%%% Theorems. (Copied from www.mitschriften.ethz.ch template)
\newtheorem{expl}{Example}[chapter]
\newtheorem{remk}{Remark}[chapter]
\newtheorem{corl}{Corollary}[chapter]
\newtheorem{thrm}{Theorem}[chapter]
\newtheorem{exer}{Exercise}[chapter]
%%% calculus: (Part of what I found some time ago somewhere in the internet.)
\newcommand{\ddd}[2]{\frac{\mathrm d ^{2} #1}{\mathrm d #2 ^{2}}}
\newcommand{\dd}[2]{\frac{\mathrm d #1}{\mathrm d #2}}
\newcommand{\ddn}[3]{\frac{\mathrm d ^{#1} #2}{\mathrm d #3 ^{#1}}}
\newcommand{\pdd}[2]{\frac{\partial #1}{\partial #2}}
\newcommand{\pddd}[2]{\frac{\partial^{2} #1}{\partial #2 ^{2}}}
\newcommand{\pdddm}[3]{\frac{\partial^{2} #1}{\partial #2 \partial #3}}
\newcommand{\pddn}[3]{\frac{\partial^{#1} #2}{\partial #3 ^{#1}}}




%\title{\boldmath The \LaTeX for physicits project}
\title{Millop improves \LaTeX~ level of physicists}


\author{Marc Maetz\\
  Institute of \LaTeX, \\ 
  ETH Zurich,\\
  8093 Zurich, Switzerland\\
  E-mail: {\em mmaetz@student.ethz.ch}}

\begin{document} 
%%% Color settings and plot settings. If the pictures look bad look at this file!

\maketitle


%\begin{abstract}
%The subject of the course an  introduction to  quantum 
%field theory. The following topics are discussed: 
%\begin{itemize}
%\item Theory of classical fields.  
%\item Canonical quantization of free fields.
%\item The  Dirac equation and  quantization of the Dirac field
%\item Field Propagation, interacting fields and perturbation theory. 
%\item Cross-sections  and decay rates.
%\item Introduction to QED and the problem of infinities.
%\item One-loop renormalization of QED. 
%\end{itemize}
%\end{abstract}

\tableofcontents 
%\listofchanges
\bibliographystyle{JHEP}
\begin{thebibliography}{10}
%\bibitem{srednicki}
%Modern Quantum Mechanics, Sakurai 
%\bibitem{sterman}
%The Feynman Lectures in Physics, Feynman 
\end{thebibliography} 
\chapter*{Introduction}
The idea is to have a \LaTeX-documentation for the whole departement of physics at the ETH Zürich. If this works well, maybe D-MATH or D-CHAB will follow etc. As a student I don't know how this is handled within the groups but all scripts given to me to study with have not been written with the best and/or the most modern \LaTeX-style. Also there are professors that don't do at all their scripts with \LaTeX. So I'd like this to change.

\LaTeX ~is really nice but whenever one wants something or to improve something, one has to google a long time through stuff that doesn't work or isn't compatible with each other. So far, I haven't found any satisfying documentation for \LaTeX ~for physicists so I will start from zero. This document should contain as few as possible solutions but always the best known one compatible with standard packages as AMSmath, etc. Whenever a package is added, it should be documented why.

This should be built up in chapters that everybody uses like maths (integrals etc.) and chapters specific to research groups.
%So far invited a bunch of students and PHD students from different institutes. Everybody that is or has been in the D-PHYS at the ETH Zürich is welcome to contribute to this project. (Of course, if this becomes an interdisciplinary project, this would be even better but for the beginning let's try something ``small''.)

\chapter{Maths}
\section{Equation formats}
\section{align}
Align to the $=$
\begin{align*}
	a&=b\\
	\leadsto caoeuuu&=d
\end{align*}
Something like this looks confusing
\begin{align*}
	&a=b\\
	&\leadsto aoeuou=d
\end{align*}
\section{Integrals}
Needs fix. Need to put the right negative space to make it look right. Put a grid in background in some way.
\subsection{Simple integral}
\begin{dgroup}[]
	\begin{dmath}[]
		\int_{}^{}\td x \, x=x^{2}
	\end{dmath}
	\begin{dmath}[]
		\int_{0}^{1}\! \td x\, x=1
	\end{dmath}
	\begin{dmath}[]
		\int_{}^{}\td x \, x=x^{2}
	\end{dmath}
	\begin{dmath}[]
		\int_{0}^{10}\! \td x\, x=100
	\end{dmath}
	\begin{dmath}[]
		\int_{}^{}\td x \, x=x^{2}
	\end{dmath}
	\begin{dmath}[]
		\int_{0}^{100}\! \td x\, x=100
	\end{dmath}
\end{dgroup}
\newpage
\subsection{With fractions}
Use thick space between two roman variables or between a roman variable and a fraction.
\begin{dgroup*}[]
	\begin{dmath}[]
		\int_{}^{}\td^{3}\mathbf{r}\;\mathbf{j}\left( \mathbf{r},t \right)=\int_{}^{}\td^{3}\mathbf{r}\; \frac{1}{2m}\left[ \psi^{*}\left( -i\hbar\nabla \right)\psi+\psi\left( i\hbar\nabla \right)\psi^{*} \right]
	\end{dmath}
	%\begin{dmath}[]
	%	\delimitershortfall=-1pt
	%	\int_{}^{}\td^{3}\mathbf{r}\;\mathbf{j}\left( \mathbf{r},t \right)=\int_{}^{}\td^{3}\mathbf{r}\; \frac{1}{2m}\left[ \psi^{*}\left( -i\hbar\nabla \right)\psi+\psi\left( i\hbar\nabla \right)\psi^{*} \right]
	%\end{dmath}
	\begin{dmath}[]
			\int_{}^{}\td^{3}\mathbf{r}\;\mathbf{j}\left( \mathbf{r},t \right)=\int_{}^{}\td^{3}\mathbf{r}\; \frac{1}{2m}\left[ \psi^{*}\left( -i\hbar\nabla \right)\psi+\psi\left( i\hbar\nabla \right)\psi^{*} \right]
	\end{dmath}
\end{dgroup*}

\subsection{Multiple integrals}
\begin{dmath}[]
	\iiint\limits_{V}\td V\,\boldsymbol{\mathrm\nabla}\cdot F = \iint\limits_{S}\td S\, F
\end{dmath}
\section{Derivatives}
\begin{dgroup}[]
	\begin{dmath}[]
		\dd{}{x}x^{2}=x
	\end{dmath}
\end{dgroup}
\section{Functions}
Examples of separations and their importance.
\begin{dgroup}[]
	\begin{dmath}[]
		\sin 2\pi\cos\theta = 0
	\end{dmath}
	\begin{dmath}[compact]
		\sin\left( \theta x \right)=\sin\theta x\neq \sin\theta \, x=\sin\left( \theta \right)x
	\end{dmath}
\end{dgroup}
\section{Multiplications}
\begin{dgroup}[]
	\begin{dmath}[]
		1\cdot 1
	\end{dmath}
	\begin{dmath}[]
		1 \! \cdot\! 1
	\end{dmath}
	\begin{dmath}[]
		1\cdot 1
	\end{dmath}
	\begin{dmath}[]
		0\cdot 0
	\end{dmath}
	\begin{dmath}[]
		x\cdot x
	\end{dmath}
	\begin{dmath}[]
		\boldsymbol{\mathrm\nabla} \cdot x
	\end{dmath}
	\begin{dmath}[]
		x \!\cdot\! x
	\end{dmath}
	\begin{dmath}[]
		\boldsymbol{\mathrm\nabla}\!\cdot\! x
	\end{dmath}
\end{dgroup}
\section{Hyphen}
\section{Braces}
I recommend using dynamic sized braces like $\left(  \right)$. However when two nested braces have the same size the readability is increased if the exterior brace has a bigger size. It is very important if the nested braces are just next to each other. One possibility is to add a vphantom and don't forget to put a $\!$ if needed.
\begin{align}
 \int \td^3 \vec r \; \vec j(\vec r , t)  &= \int \td^3\vec r \; \frac{1}{2m} \left[ \!\vphantom{A^{A}}
 \braket{\psi, t }{\vec r} \sand{\vec r}{\vec p}{ \psi, t}
 + 
 \sand{ \psi, t}{\vec p}{\vec r} \braket {\vec r}{\psi, t }
 \right] 
\end{align} 
Without the vphantom:
\begin{align}
 \int \td^3 \vec r \; \vec j(\vec r , t)  &= \int \td^3\vec r \; \frac{1}{2m} \left[ 
 \braket{\psi, t }{\vec r} \sand{\vec r}{\vec p}{ \psi, t}
 + 
 \sand{ \psi, t}{\vec p}{\vec r} \braket {\vec r}{\psi, t }
 \right] 
\end{align} 
Alternative with %\delimitershortfall=-1pt and \mathinner


\section{Prime}
$A'$ and $A^{\prime}$ look exactly the same.
\section{Square root}
\section{Matrix}
When there is an underbrace under the matrix, one needs a $\!$ on both sides
\begin{dgroup*}[]
	\begin{dmath}[]
		\underbrace{\!
			\begin{pmatrix}
				1&2\\
				3&4
			\end{pmatrix}
		\!}_{blabla}
		=
		\begin{pmatrix}
			5&6\\
			7&8
		\end{pmatrix}
	\end{dmath}
	\begin{dmath}[]
		\underbrace{
			\begin{pmatrix}
				1&2\\
				3&4
			\end{pmatrix}
		}_{blabla}
		=
		\begin{pmatrix}
			5&6\\
			7&8
		\end{pmatrix}
	\end{dmath}
\end{dgroup*}
\begin{dmath}[]
	\begin{pmatrix}
		\hdotsfor{3}\\
		\ldots&1&\ldots\\
		\hdotsfor{3}
	\end{pmatrix}
\end{dmath}
\section{Miscellaneous}
\begin{enumerate}[i)]
	\item Use $\ell$ instead of $l$. Increases readability.
	\item $\xrightarrow{T}$ instead of $\stackrel{T}{\rightarrow}$
	\item $\ddots$
\end{enumerate}

  
\input{templates.tex}  
\chapter{Useful packages}
\section{pgf/tikz}
\section{change}

\newpage

%%%%%%%%%%%%%%%%%%%%%%%%%%%%%%%%%%%%%%%%%%%%%%%%%
%%%%% bibliography
%%%%%%%%%%%%%%%%%%%%%%%%%%%%%%%%%%%%%%%%%%%%%%%%%%


\end{document}

% LocalWords:  asimple
